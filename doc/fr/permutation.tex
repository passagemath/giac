\documentclass{article}
\usepackage[latin1]{inputenc}
\usepackage[T1]{fontenc}
\usepackage[francais]{babel}
\usepackage{amsfonts}
\begin{document}
\title{Groupes de permutations}
\author{Rene� De Graeve}

\section{Les th�or�mes} \label{sec:theo}
Soit  ${\tt J_n=[0,1..n-1]}$.
Une permutation {\tt p} de {\tt n} �l�ments est une application bijective de 
${\tt J_n}$ dans lui m\^eme et induit une application ${\tt \pi_p}$ de  
${\tt \mathbb Z^n}$ dans lui m\^eme (${\tt \pi_p(x_i)=x_{p(i)}}$).\\
Les permutations forment un groupe pour la composition des applications.\\
Un cycle {\tt c} est une permutation telle qu'il existe un enter  
$k \ (0 \leq k \leq n-1)$ v�rifiant :\\
pour $j=0...k-1 \ \ c(a_j)=a_{j+1}$ et $c(a_k)=a_0$\\
pour $j=k+1...n-1 \ \ c(a_j)=a_j$\\
$k$ est appel� l'ordre du cycle $c$ ($c^k=id$).\\
Un cycle d'ordre $k$ est aussi appel� une permutation circulaire d'ordre $k$.\\
Une transposition $t$ est un cycle d'ordre 2 ($t^2=id$).\\

Th�or�me 1\\
Toute permutation peut s'exprimer comme produit de cycles disjoints.\\ 
L'ordre d'une permutation est le plus petit commun multiple $k$ des ordres 
des cycles disjoints obtenus ($p^k=id$).\\

Th�or�me 2\\
Toute permutation peut s'exprimer comme produit de transpositions.\\
\section{Notations} \label{sec:notationperm}
Une permutation {\tt p} sera not�e :\\
{\tt [p(0),p(1),...,p(n-1)]}\\
Un cycle d'ordre {\tt c} $k$ sera not� :\\
${\tt [a_0,a_1,...,a_k]}$ si ${\tt c(a_0)=a_1...c(a_k)=a_0}$\\
Attention :\\
{\tt [0,2,1]} repr�sente la permutation {\tt p} telle que :\\
{\tt p(0)=0, p(1)=2, p(2)=1}\\
 mais repr�sente aussi le cycle {\tt c} tel que :\\
{\tt c(0)=2, c(2)=1, c(1)=0}
\section{Exercices} \label{sec:exoperm}

{\tt Exercice 1}\\
Exprimer les permutations suivantes comme produit de cycles disjoints et 
d�terminer leurs signatures, leurs ordres  et leurs inverses:\\
{\tt [1,2,0]}\\
{\tt [2,0,1]}\\
{\tt [2,1,0]}\\
{\tt [1,2,0,3]}\\
{\tt [1,0,3,2]}\\
{\tt [2,1,3,0]}\\
{\tt [2,4,5,0,1,3]}\\
{\tt [5,0,4,2,3,1]}\\
{\tt [5,3,4,6,2,0,1]}\\

Exercice 2\\
Ecrire les produits de cycles suivants sous la forme :\\
1/ d'une permutation\\
2/ d'un produit de cycles disjoints\\
{\tt [[0,1,2,3,4],[0,4,5],[1,3,5]]}\\
{\tt [[0,1,2,3],[1,2,3,4],[2,3,4,0]]}\\
{\tt [[0,1],[1,2],[2,3],[3,4],[4,0]]}\\
{\tt [[1,5][4,5],[5,6]]}\\

Exercice 3\\
Calculer $p^{1000}$ pour les permutations $p$ suivantes :\\
{\tt [2,5,7,8,3,4,1,0,6]}\\
{\tt [2,5,0,3,1,4]}\\

Exercice 4\\
D\'eterminer le groupe engendr\'e par les permutations suivantes :\\
{\tt [2,1,0,3]}\\
{\tt [3,1,2,0]}\\

\section{Corrections des exercices} \label{sec:corexoperm}
Exercice 1\\
Exercice 2\\
On tape :\\
{\tt cl:=[[0,1,2,3,4],[0,4,5],[1,3,5]] }\\
On tape :\\
{\tt cycles2permu(cl)}\\
On obtient :\\
{\tt [0,4,3,1,5,2]}\\
Exercice 3\\
On tape :\\
{\tt permu2cycles([2,5,7,8,3,4,1,0,6])}\\
On obtient :\\
{\tt [[0,2,7],[1,5,4,3,8,6]]}\\
On tape :\\
{\tt p:=[2,5,7,8,3,4,1,0,6]}\\
$p$ est d�composable en un cycle d''orde 3 et un cycle d'ordre 6, $p$ est donc d'ordre 6 ({\tt lcm(3,6)=6}).\\
Donc $p^6=id$. On a {\tt irem(1000,6)=4} donc :\\
$p^{1000}=p^4$\\
On tape :\\
{\tt p2:=p1op2(p,p)}\\
 {\tt p4:=p1op2(p2,p2)}\\
On obtient :\\
{\tt [2,8,7,5,1,6,3,0,4]}\\
Donc $p^{1000}=p^4=  [2,8,7,5,1,6,3,0,4]$\\
On v�rifie que $p^6=id$ :\\
{\tt p6:=p1op2(p2,p4)}
On obtient bien :\\
{\tt [0,1,2,3,4,5,6,7,8]}\\

On tape :\\
{\tt permu2cycles([2,5,0,3,1,4])}\\
On obtient :\\
{\tt [[0,2],[1,5,4]]}\\
On tape :\\
{\tt p:=[2,5,0,3,1,4]}\\
$p$ est d�composable en un cycle d''orde 2 et un cycle d'ordre 3, $p$ est donc d'ordre 6 ({\tt lcm(2,3)=6}).\\
Donc $p^6=id$. On a {\tt irem(1000,6)=4} donc :\\
$p^{1000}=p^4$\\
On tape :\_
{\tt p2:=p1op2(p,p)}\\
 {\tt p4:=p1op2(p2,p2)}\\
On obtient :\\
{\tt [0,5,2,3,1,4]}\\
Donc $p^{1000}=p^4=[0,5,2,3,1,4]$\\
On v�rifie que $p^6=id$ :\\
{\tt p6:=p1op2(p2,p4)}
On obtient bien :\\
{\tt [0,1,2,3,4,5]}\\
Exercice 4\\
On tape :\\
{\tt groupermu([2,1,0,3],[3,1,2,0])}\\
On obtient :\\
{\tt [[2,1,0,3],[3,1,2,0],[0,1,2,3],[2,1,3,0],[3,1,0,2],[0,1,3,2]]}\\
qui sont :\\
{\tt a,b,a*a,b*a,a*b,a*b*a} avec {\tt a:=[2,1,0,3] b:=[3,1,2,0]}
On peut v\'erifier par exemple que :\\
{\tt a*b*a=p1op2(a,p1op2(b,a))=b*a*b=p1op2(b,p1op2(a,b))=[0,1,3,2]}
\end{document}
